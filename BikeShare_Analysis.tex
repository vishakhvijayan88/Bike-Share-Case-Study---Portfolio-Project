% Options for packages loaded elsewhere
\PassOptionsToPackage{unicode}{hyperref}
\PassOptionsToPackage{hyphens}{url}
%
\documentclass[
]{article}
\usepackage{amsmath,amssymb}
\usepackage{lmodern}
\usepackage{iftex}
\ifPDFTeX
  \usepackage[T1]{fontenc}
  \usepackage[utf8]{inputenc}
  \usepackage{textcomp} % provide euro and other symbols
\else % if luatex or xetex
  \usepackage{unicode-math}
  \defaultfontfeatures{Scale=MatchLowercase}
  \defaultfontfeatures[\rmfamily]{Ligatures=TeX,Scale=1}
\fi
% Use upquote if available, for straight quotes in verbatim environments
\IfFileExists{upquote.sty}{\usepackage{upquote}}{}
\IfFileExists{microtype.sty}{% use microtype if available
  \usepackage[]{microtype}
  \UseMicrotypeSet[protrusion]{basicmath} % disable protrusion for tt fonts
}{}
\makeatletter
\@ifundefined{KOMAClassName}{% if non-KOMA class
  \IfFileExists{parskip.sty}{%
    \usepackage{parskip}
  }{% else
    \setlength{\parindent}{0pt}
    \setlength{\parskip}{6pt plus 2pt minus 1pt}}
}{% if KOMA class
  \KOMAoptions{parskip=half}}
\makeatother
\usepackage{xcolor}
\usepackage[margin=1in]{geometry}
\usepackage{color}
\usepackage{fancyvrb}
\newcommand{\VerbBar}{|}
\newcommand{\VERB}{\Verb[commandchars=\\\{\}]}
\DefineVerbatimEnvironment{Highlighting}{Verbatim}{commandchars=\\\{\}}
% Add ',fontsize=\small' for more characters per line
\usepackage{framed}
\definecolor{shadecolor}{RGB}{248,248,248}
\newenvironment{Shaded}{\begin{snugshade}}{\end{snugshade}}
\newcommand{\AlertTok}[1]{\textcolor[rgb]{0.94,0.16,0.16}{#1}}
\newcommand{\AnnotationTok}[1]{\textcolor[rgb]{0.56,0.35,0.01}{\textbf{\textit{#1}}}}
\newcommand{\AttributeTok}[1]{\textcolor[rgb]{0.77,0.63,0.00}{#1}}
\newcommand{\BaseNTok}[1]{\textcolor[rgb]{0.00,0.00,0.81}{#1}}
\newcommand{\BuiltInTok}[1]{#1}
\newcommand{\CharTok}[1]{\textcolor[rgb]{0.31,0.60,0.02}{#1}}
\newcommand{\CommentTok}[1]{\textcolor[rgb]{0.56,0.35,0.01}{\textit{#1}}}
\newcommand{\CommentVarTok}[1]{\textcolor[rgb]{0.56,0.35,0.01}{\textbf{\textit{#1}}}}
\newcommand{\ConstantTok}[1]{\textcolor[rgb]{0.00,0.00,0.00}{#1}}
\newcommand{\ControlFlowTok}[1]{\textcolor[rgb]{0.13,0.29,0.53}{\textbf{#1}}}
\newcommand{\DataTypeTok}[1]{\textcolor[rgb]{0.13,0.29,0.53}{#1}}
\newcommand{\DecValTok}[1]{\textcolor[rgb]{0.00,0.00,0.81}{#1}}
\newcommand{\DocumentationTok}[1]{\textcolor[rgb]{0.56,0.35,0.01}{\textbf{\textit{#1}}}}
\newcommand{\ErrorTok}[1]{\textcolor[rgb]{0.64,0.00,0.00}{\textbf{#1}}}
\newcommand{\ExtensionTok}[1]{#1}
\newcommand{\FloatTok}[1]{\textcolor[rgb]{0.00,0.00,0.81}{#1}}
\newcommand{\FunctionTok}[1]{\textcolor[rgb]{0.00,0.00,0.00}{#1}}
\newcommand{\ImportTok}[1]{#1}
\newcommand{\InformationTok}[1]{\textcolor[rgb]{0.56,0.35,0.01}{\textbf{\textit{#1}}}}
\newcommand{\KeywordTok}[1]{\textcolor[rgb]{0.13,0.29,0.53}{\textbf{#1}}}
\newcommand{\NormalTok}[1]{#1}
\newcommand{\OperatorTok}[1]{\textcolor[rgb]{0.81,0.36,0.00}{\textbf{#1}}}
\newcommand{\OtherTok}[1]{\textcolor[rgb]{0.56,0.35,0.01}{#1}}
\newcommand{\PreprocessorTok}[1]{\textcolor[rgb]{0.56,0.35,0.01}{\textit{#1}}}
\newcommand{\RegionMarkerTok}[1]{#1}
\newcommand{\SpecialCharTok}[1]{\textcolor[rgb]{0.00,0.00,0.00}{#1}}
\newcommand{\SpecialStringTok}[1]{\textcolor[rgb]{0.31,0.60,0.02}{#1}}
\newcommand{\StringTok}[1]{\textcolor[rgb]{0.31,0.60,0.02}{#1}}
\newcommand{\VariableTok}[1]{\textcolor[rgb]{0.00,0.00,0.00}{#1}}
\newcommand{\VerbatimStringTok}[1]{\textcolor[rgb]{0.31,0.60,0.02}{#1}}
\newcommand{\WarningTok}[1]{\textcolor[rgb]{0.56,0.35,0.01}{\textbf{\textit{#1}}}}
\usepackage{graphicx}
\makeatletter
\def\maxwidth{\ifdim\Gin@nat@width>\linewidth\linewidth\else\Gin@nat@width\fi}
\def\maxheight{\ifdim\Gin@nat@height>\textheight\textheight\else\Gin@nat@height\fi}
\makeatother
% Scale images if necessary, so that they will not overflow the page
% margins by default, and it is still possible to overwrite the defaults
% using explicit options in \includegraphics[width, height, ...]{}
\setkeys{Gin}{width=\maxwidth,height=\maxheight,keepaspectratio}
% Set default figure placement to htbp
\makeatletter
\def\fps@figure{htbp}
\makeatother
\setlength{\emergencystretch}{3em} % prevent overfull lines
\providecommand{\tightlist}{%
  \setlength{\itemsep}{0pt}\setlength{\parskip}{0pt}}
\setcounter{secnumdepth}{-\maxdimen} % remove section numbering
\ifLuaTeX
  \usepackage{selnolig}  % disable illegal ligatures
\fi
\IfFileExists{bookmark.sty}{\usepackage{bookmark}}{\usepackage{hyperref}}
\IfFileExists{xurl.sty}{\usepackage{xurl}}{} % add URL line breaks if available
\urlstyle{same} % disable monospaced font for URLs
\hypersetup{
  pdftitle={BikeShare Analysis},
  pdfauthor={Vishakh Vijayan},
  hidelinks,
  pdfcreator={LaTeX via pandoc}}

\title{BikeShare Analysis}
\author{Vishakh Vijayan}
\date{2022-10-08}

\begin{document}
\maketitle

\hypertarget{bikeshare-analysis}{%
\section{BikeShare Analysis}\label{bikeshare-analysis}}

\hypertarget{introduction}{%
\subsection{Introduction}\label{introduction}}

\hypertarget{data-cleaning-transforming-and-wrangling}{%
\subsubsection{Data cleaning, transforming and
wrangling}\label{data-cleaning-transforming-and-wrangling}}

This report shows step-by-step analysis of Bike-Share data of hypothetic
Chicago Bike Share Company. The data is hosted in the link here and
publicly available:
\url{https://divvy-tripdata.s3.amazonaws.com/index.html}. The required
analysis is for last 12 months and hence this data is downloaded from
the source in .csv format.

Due to large data sets, the files are more than 100mb which caused
difficulty to work with in cleaning and transforming. Hence decided to
do this step in Google BigQuery using SQL Query, but BigQuery sandbox
has a file size limit to upload into the database. For this, they were
converted to .xls format to decrease the file size and then separated
the longitude and latitude data into separate .csv files and remaining
columns for 12 months in separate .csv files. Some initial cleaning up
were done in Excel Power Query to identify null values in ride\_id and
dates. Null values in stations were kept unless the latitude and
longitude are also nulls. This is assuming that, rides can be start and
end any where not necessarily a station. Also, if duration of the ride
is zero, they were also removed from the tables. After cleaning,
separating and saving as .csv, all files were now below 100mb and
uploaded into tables in a database in BigQuery.

After uploading the files, files were merged into single raw data table
after combining inidividual monthly tables and lon-lat tables using
union and joins and saved as view. Any further cleaning was done based
on contexts and analyzed in detail using SQL Query. The cleaned raw data
is downloaded for analysis in R and saved in working directory.

\hypertarget{data-visualization-and-further-analysis}{%
\subsubsection{Data Visualization and further
analysis}\label{data-visualization-and-further-analysis}}

For further analysis we need to create visualizations to understand the
rides taking place across the last 12 months. To do this, we need to
first install and load the required packages.

\begin{Shaded}
\begin{Highlighting}[]
\FunctionTok{library}\NormalTok{(tidyverse)}
\end{Highlighting}
\end{Shaded}

\begin{verbatim}
## -- Attaching packages --------------------------------------- tidyverse 1.3.2 --
## v ggplot2 3.3.6      v purrr   0.3.4 
## v tibble  3.1.8      v dplyr   1.0.10
## v tidyr   1.2.1      v stringr 1.4.1 
## v readr   2.1.3      v forcats 0.5.2 
## -- Conflicts ------------------------------------------ tidyverse_conflicts() --
## x dplyr::filter() masks stats::filter()
## x dplyr::lag()    masks stats::lag()
\end{verbatim}

\begin{Shaded}
\begin{Highlighting}[]
\FunctionTok{library}\NormalTok{(lubridate)}
\end{Highlighting}
\end{Shaded}

\begin{verbatim}
## 
## Attaching package: 'lubridate'
## 
## The following objects are masked from 'package:base':
## 
##     date, intersect, setdiff, union
\end{verbatim}

\begin{Shaded}
\begin{Highlighting}[]
\FunctionTok{library}\NormalTok{(dplyr)}
\end{Highlighting}
\end{Shaded}

Next let's import the cleaned dataset which has been downloaded from
bigquery and stored in the local working directory.

\begin{Shaded}
\begin{Highlighting}[]
\NormalTok{BikeShareData }\OtherTok{\textless{}{-}} \FunctionTok{read\_csv}\NormalTok{(}\StringTok{"BikeShareData\_Clean.csv"}\NormalTok{)}
\end{Highlighting}
\end{Shaded}

\begin{verbatim}
## Rows: 5319274 Columns: 16
## -- Column specification --------------------------------------------------------
## Delimiter: ","
## chr (9): ride_id, started_at, ended_at, member_casual, rideable_type, start_...
## dbl (7): duration_minutes, duration_hours, duration_days, start_lat, start_l...
## 
## i Use `spec()` to retrieve the full column specification for this data.
## i Specify the column types or set `show_col_types = FALSE` to quiet this message.
\end{verbatim}

\begin{Shaded}
\begin{Highlighting}[]
\FunctionTok{head}\NormalTok{(BikeShareData)}
\end{Highlighting}
\end{Shaded}

\begin{verbatim}
## # A tibble: 6 x 16
##   ride_id        start~1 ended~2 durat~3 durat~4 durat~5 membe~6 ridea~7 start~8
##   <chr>          <chr>   <chr>     <dbl>   <dbl>   <dbl> <chr>   <chr>   <chr>  
## 1 46719C2D3D513~ 2021-0~ 2021-0~       1    0.02       0 casual  classi~ TA1307~
## 2 F3A82C6371C22~ 2022-0~ 2022-0~       1    0.02       0 member  classi~ TA1307~
## 3 2FC8CF798C7C5~ 2022-0~ 2022-0~       1    0.02       0 member  electr~ <NA>   
## 4 74ADAA803CCBE~ 2022-0~ 2022-0~       1    0.02       0 member  electr~ 13021  
## 5 C246B15EA73DE~ 2021-0~ 2021-0~       1    0.02       0 member  classi~ TA1306~
## 6 08AB194827CCC~ 2021-0~ 2021-0~       1    0.02       0 member  electr~ 13197  
## # ... with 7 more variables: start_station_name <chr>, end_station_id <chr>,
## #   end_station_name <chr>, start_lat <dbl>, start_lng <dbl>, end_lat <dbl>,
## #   end_lng <dbl>, and abbreviated variable names 1: started_at, 2: ended_at,
## #   3: duration_minutes, 4: duration_hours, 5: duration_days, 6: member_casual,
## #   7: rideable_type, 8: start_station_id
\end{verbatim}

The initial data shows all the columns imported correctly. Let's look at
the data types and overview of the dataframe.

\begin{Shaded}
\begin{Highlighting}[]
\FunctionTok{str}\NormalTok{(BikeShareData)}
\end{Highlighting}
\end{Shaded}

\begin{verbatim}
## spec_tbl_df [5,319,274 x 16] (S3: spec_tbl_df/tbl_df/tbl/data.frame)
##  $ ride_id           : chr [1:5319274] "46719C2D3D513D99" "F3A82C6371C22F1D" "2FC8CF798C7C57AF" "74ADAA803CCBE929" ...
##  $ started_at        : chr [1:5319274] "2021-09-26 10:35:00 UTC" "2022-06-14 18:06:00 UTC" "2022-05-29 13:46:00 UTC" "2022-08-25 08:54:00 UTC" ...
##  $ ended_at          : chr [1:5319274] "2021-09-26 10:36:00 UTC" "2022-06-14 18:07:00 UTC" "2022-05-29 13:47:00 UTC" "2022-08-25 08:55:00 UTC" ...
##  $ duration_minutes  : num [1:5319274] 1 1 1 1 1 1 1 1 1 1 ...
##  $ duration_hours    : num [1:5319274] 0.02 0.02 0.02 0.02 0.02 0.02 0.02 0.02 0.02 0.02 ...
##  $ duration_days     : num [1:5319274] 0 0 0 0 0 0 0 0 0 0 ...
##  $ member_casual     : chr [1:5319274] "casual" "member" "member" "member" ...
##  $ rideable_type     : chr [1:5319274] "classic_bike" "classic_bike" "electric_bike" "electric_bike" ...
##  $ start_station_id  : chr [1:5319274] "TA1307000150" "TA1307000062" NA "13021" ...
##  $ start_station_name: chr [1:5319274] "Pine Grove Ave & Waveland Ave" "Sedgwick St & Huron St" NA "Clinton St & Lake St" ...
##  $ end_station_id    : chr [1:5319274] "TA1307000150" "TA1307000062" "WL-012" "WL-012" ...
##  $ end_station_name  : chr [1:5319274] "Pine Grove Ave & Waveland Ave" "Sedgwick St & Huron St" "Clinton St & Washington Blvd" "Clinton St & Washington Blvd" ...
##  $ start_lat         : num [1:5319274] 41.9 41.9 41.9 41.9 41.9 ...
##  $ start_lng         : num [1:5319274] -87.6 -87.6 -87.6 -87.6 -87.7 ...
##  $ end_lat           : num [1:5319274] 41.9 41.9 41.9 41.9 41.9 ...
##  $ end_lng           : num [1:5319274] -87.6 -87.6 -87.6 -87.6 -87.7 ...
##  - attr(*, "spec")=
##   .. cols(
##   ..   ride_id = col_character(),
##   ..   started_at = col_character(),
##   ..   ended_at = col_character(),
##   ..   duration_minutes = col_double(),
##   ..   duration_hours = col_double(),
##   ..   duration_days = col_double(),
##   ..   member_casual = col_character(),
##   ..   rideable_type = col_character(),
##   ..   start_station_id = col_character(),
##   ..   start_station_name = col_character(),
##   ..   end_station_id = col_character(),
##   ..   end_station_name = col_character(),
##   ..   start_lat = col_double(),
##   ..   start_lng = col_double(),
##   ..   end_lat = col_double(),
##   ..   end_lng = col_double()
##   .. )
##  - attr(*, "problems")=<externalptr>
\end{verbatim}

From this we found that started\_at and ended\_at columns are stored as
character data types. We need to change them into date data type. For
this, we will first extract the first 10 date characters into new
columns and then convert them into date columns using mutate function.

\begin{Shaded}
\begin{Highlighting}[]
\NormalTok{RidesLast12Months }\OtherTok{\textless{}{-}}\NormalTok{ BikeShareData }\SpecialCharTok{\%\textgreater{}\%}
  \FunctionTok{select}\NormalTok{(ride\_id, member\_casual, rideable\_type, started\_at, ended\_at)}

\NormalTok{RidesLast12Months }\OtherTok{\textless{}{-}}\NormalTok{ RidesLast12Months }\SpecialCharTok{\%\textgreater{}\%}
  \FunctionTok{transform}\NormalTok{(}\AttributeTok{started\_at =} \FunctionTok{substr}\NormalTok{(started\_at, }\DecValTok{1}\NormalTok{, }\DecValTok{10}\NormalTok{), }\AttributeTok{ended\_at =} \FunctionTok{substr}\NormalTok{(ended\_at, }\DecValTok{1}\NormalTok{, }\DecValTok{10}\NormalTok{))}

\NormalTok{RidesLast12Months}\SpecialCharTok{$}\NormalTok{started\_at }\OtherTok{\textless{}{-}} \FunctionTok{ymd}\NormalTok{(RidesLast12Months}\SpecialCharTok{$}\NormalTok{started\_at)}
\NormalTok{RidesLast12Months}\SpecialCharTok{$}\NormalTok{ended\_at }\OtherTok{\textless{}{-}} \FunctionTok{ymd}\NormalTok{(RidesLast12Months}\SpecialCharTok{$}\NormalTok{ended\_at)}

\FunctionTok{head}\NormalTok{(RidesLast12Months)}
\end{Highlighting}
\end{Shaded}

\begin{verbatim}
##            ride_id member_casual rideable_type started_at   ended_at
## 1 46719C2D3D513D99        casual  classic_bike 2021-09-26 2021-09-26
## 2 F3A82C6371C22F1D        member  classic_bike 2022-06-14 2022-06-14
## 3 2FC8CF798C7C57AF        member electric_bike 2022-05-29 2022-05-29
## 4 74ADAA803CCBE929        member electric_bike 2022-08-25 2022-08-25
## 5 C246B15EA73DEAF6        member  classic_bike 2021-09-26 2021-09-26
## 6 08AB194827CCC83E        member electric_bike 2021-09-13 2021-09-13
\end{verbatim}

We have transformed the started\_at and ended\_at columns to date
datatype. Now let's look at how the ride counts look like throughout
last 12 months. For this we will use the ride start dates and see the
count of rides throughout the year.

\begin{Shaded}
\begin{Highlighting}[]
\NormalTok{RideCounts\_Last12Months }\OtherTok{\textless{}{-}}\NormalTok{ RidesLast12Months }\SpecialCharTok{\%\textgreater{}\%}
  \FunctionTok{select}\NormalTok{(ride\_id, member\_casual, started\_at) }\SpecialCharTok{\%\textgreater{}\%}
  \FunctionTok{group\_by}\NormalTok{(member\_casual, started\_at) }\SpecialCharTok{\%\textgreater{}\%}
  \FunctionTok{summarize}\NormalTok{(}\AttributeTok{ride\_count =} \FunctionTok{n}\NormalTok{())}
\end{Highlighting}
\end{Shaded}

\begin{verbatim}
## `summarise()` has grouped output by 'member_casual'. You can override using the
## `.groups` argument.
\end{verbatim}

\begin{Shaded}
\begin{Highlighting}[]
\FunctionTok{head}\NormalTok{(RideCounts\_Last12Months)}
\end{Highlighting}
\end{Shaded}

\begin{verbatim}
## # A tibble: 6 x 3
## # Groups:   member_casual [1]
##   member_casual started_at ride_count
##   <chr>         <date>          <int>
## 1 casual        2021-09-01       9317
## 2 casual        2021-09-02       9980
## 3 casual        2021-09-03      10816
## 4 casual        2021-09-04      15920
## 5 casual        2021-09-05      20375
## 6 casual        2021-09-06      16787
\end{verbatim}

We have created a data frame with summary of ride counts by dates and
ride types

Now let's visualize how the rides are distributed throughout last 12
month period to see if there is any pattern in the rides.

\begin{Shaded}
\begin{Highlighting}[]
\FunctionTok{ggplot}\NormalTok{(RideCounts\_Last12Months)}\SpecialCharTok{+}\FunctionTok{geom\_point}\NormalTok{(}\AttributeTok{mapping =} \FunctionTok{aes}\NormalTok{(}\AttributeTok{x =}\NormalTok{ started\_at, }\AttributeTok{y =}\NormalTok{ ride\_count, }\AttributeTok{color =}\NormalTok{ member\_casual)) }\SpecialCharTok{+} \FunctionTok{geom\_smooth}\NormalTok{(}\AttributeTok{method =} \StringTok{"gam"}\NormalTok{, }\AttributeTok{mapping =} \FunctionTok{aes}\NormalTok{(}\AttributeTok{x =}\NormalTok{ started\_at, }\AttributeTok{y =}\NormalTok{ ride\_count)) }\SpecialCharTok{+} \FunctionTok{labs}\NormalTok{(}\AttributeTok{title =} \StringTok{"Bike Share: Rides in Last 12 Months"}\NormalTok{, }\AttributeTok{subtitle =} \StringTok{"Comparison of Casual and Member rides"}\NormalTok{)  }
\end{Highlighting}
\end{Shaded}

\begin{verbatim}
## `geom_smooth()` using formula 'y ~ s(x, bs = "cs")'
\end{verbatim}

\includegraphics{BikeShare_Analysis_files/figure-latex/unnamed-chunk-6-1.pdf}

From the scatter-plot, there is a very clear pattern that riders during
winter times end of October through end of May are significantly less.
So from this it looks like instead of an yearly pass, people will be
much more attracted towards a 6 month membership.

\begin{Shaded}
\begin{Highlighting}[]
\FunctionTok{ggplot}\NormalTok{(RideCounts\_Last12Months)}\SpecialCharTok{+}\FunctionTok{geom\_point}\NormalTok{(}\AttributeTok{mapping =} \FunctionTok{aes}\NormalTok{(}\AttributeTok{x =}\NormalTok{ started\_at, }\AttributeTok{y =}\NormalTok{ ride\_count, }\AttributeTok{color =}\NormalTok{ member\_casual)) }\SpecialCharTok{+} \FunctionTok{geom\_smooth}\NormalTok{(}\AttributeTok{method =} \StringTok{"gam"}\NormalTok{, }\AttributeTok{mapping =} \FunctionTok{aes}\NormalTok{(}\AttributeTok{x =}\NormalTok{ started\_at, }\AttributeTok{y =}\NormalTok{ ride\_count)) }\SpecialCharTok{+} \FunctionTok{facet\_wrap}\NormalTok{(}\SpecialCharTok{\textasciitilde{}}\NormalTok{member\_casual) }\SpecialCharTok{+} \FunctionTok{labs}\NormalTok{(}\AttributeTok{title =} \StringTok{"Bike Share: Rides in last 12 Months"}\NormalTok{, }\AttributeTok{subtitle =} \StringTok{"Casual Rides and Member Rides through last 12 months"}\NormalTok{)}
\end{Highlighting}
\end{Shaded}

\begin{verbatim}
## `geom_smooth()` using formula 'y ~ s(x, bs = "cs")'
\end{verbatim}

\includegraphics{BikeShare_Analysis_files/figure-latex/unnamed-chunk-7-1.pdf}

From this, we can see that there is a small dip in casual riders in
summer, after the initial peak. But the member riders were quite
consistently high during this period. Now let's see how the same data by
ride types.

\begin{Shaded}
\begin{Highlighting}[]
\NormalTok{RideCounts\_Last12Months }\OtherTok{\textless{}{-}}\NormalTok{ RidesLast12Months }\SpecialCharTok{\%\textgreater{}\%}
  \FunctionTok{select}\NormalTok{(ride\_id, member\_casual, rideable\_type, started\_at) }\SpecialCharTok{\%\textgreater{}\%}
  \FunctionTok{group\_by}\NormalTok{(member\_casual, rideable\_type, started\_at) }\SpecialCharTok{\%\textgreater{}\%}
  \FunctionTok{summarize}\NormalTok{(}\AttributeTok{ride\_count =} \FunctionTok{n}\NormalTok{())}
\end{Highlighting}
\end{Shaded}

\begin{verbatim}
## `summarise()` has grouped output by 'member_casual', 'rideable_type'. You can
## override using the `.groups` argument.
\end{verbatim}

\begin{Shaded}
\begin{Highlighting}[]
\FunctionTok{head}\NormalTok{(RideCounts\_Last12Months)}
\end{Highlighting}
\end{Shaded}

\begin{verbatim}
## # A tibble: 6 x 4
## # Groups:   member_casual, rideable_type [1]
##   member_casual rideable_type started_at ride_count
##   <chr>         <chr>         <date>          <int>
## 1 casual        classic_bike  2021-09-01       5681
## 2 casual        classic_bike  2021-09-02       5879
## 3 casual        classic_bike  2021-09-03       6047
## 4 casual        classic_bike  2021-09-04       9733
## 5 casual        classic_bike  2021-09-05      12834
## 6 casual        classic_bike  2021-09-06      10474
\end{verbatim}

\begin{Shaded}
\begin{Highlighting}[]
\FunctionTok{ggplot}\NormalTok{(RideCounts\_Last12Months) }\SpecialCharTok{+} \FunctionTok{geom\_point}\NormalTok{(}\AttributeTok{mapping =} \FunctionTok{aes}\NormalTok{(}\AttributeTok{x =}\NormalTok{ started\_at, }\AttributeTok{y =}\NormalTok{ ride\_count, }\AttributeTok{color =}\NormalTok{ rideable\_type)) }\SpecialCharTok{+} \FunctionTok{geom\_smooth}\NormalTok{(}\AttributeTok{method =} \StringTok{"gam"}\NormalTok{, }\AttributeTok{mapping =} \FunctionTok{aes}\NormalTok{(}\AttributeTok{x =}\NormalTok{ started\_at, }\AttributeTok{y =}\NormalTok{ ride\_count)) }\SpecialCharTok{+} \FunctionTok{facet\_wrap}\NormalTok{(}\SpecialCharTok{\textasciitilde{}}\NormalTok{member\_casual) }\SpecialCharTok{+} \FunctionTok{labs}\NormalTok{(}\AttributeTok{title =} \StringTok{"Bike Share: Ride Counts by Ride Types and rider type"}\NormalTok{, }\AttributeTok{subtitle =} \StringTok{"Shows the difference in rider behavior by bike type"}\NormalTok{)}
\end{Highlighting}
\end{Shaded}

\begin{verbatim}
## `geom_smooth()` using formula 'y ~ s(x, bs = "cs")'
\end{verbatim}

\includegraphics{BikeShare_Analysis_files/figure-latex/unnamed-chunk-8-1.pdf}

Let's look at the data in a different way, using the bar charts.

\begin{Shaded}
\begin{Highlighting}[]
\NormalTok{RideCounts\_Last12Months }\OtherTok{\textless{}{-}}\NormalTok{ RidesLast12Months }\SpecialCharTok{\%\textgreater{}\%}
  \FunctionTok{select}\NormalTok{(ride\_id, member\_casual, rideable\_type, started\_at)}

\FunctionTok{ggplot}\NormalTok{(RideCounts\_Last12Months) }\SpecialCharTok{+} \FunctionTok{geom\_bar}\NormalTok{(}\AttributeTok{mapping =} \FunctionTok{aes}\NormalTok{(}\AttributeTok{x =}\NormalTok{ rideable\_type, }\AttributeTok{fill =}\NormalTok{ rideable\_type)) }\SpecialCharTok{+} \FunctionTok{facet\_wrap}\NormalTok{(}\SpecialCharTok{\textasciitilde{}}\NormalTok{member\_casual) }\SpecialCharTok{+} \FunctionTok{labs}\NormalTok{(}\AttributeTok{title =} \StringTok{"Bike Share: Ride count by rider type and bike type"}\NormalTok{)}
\end{Highlighting}
\end{Shaded}

\includegraphics{BikeShare_Analysis_files/figure-latex/unnamed-chunk-9-1.pdf}

Let's see the ride counts in specific periods: Sep - Mar and Apr - Aug
First let us look date range Sep - Mar

\begin{Shaded}
\begin{Highlighting}[]
\NormalTok{RideCounts\_SepMar }\OtherTok{\textless{}{-}}\NormalTok{ RidesLast12Months }\SpecialCharTok{\%\textgreater{}\%}
  \FunctionTok{select}\NormalTok{(ride\_id, member\_casual, rideable\_type, started\_at) }\SpecialCharTok{\%\textgreater{}\%}
  \FunctionTok{filter}\NormalTok{(started\_at }\SpecialCharTok{\textgreater{}=} \StringTok{\textquotesingle{}2021{-}09{-}01\textquotesingle{}}\NormalTok{, started\_at }\SpecialCharTok{\textless{}=} \StringTok{\textquotesingle{}2022{-}03{-}31\textquotesingle{}}\NormalTok{)}

\FunctionTok{ggplot}\NormalTok{(RideCounts\_SepMar) }\SpecialCharTok{+} \FunctionTok{geom\_bar}\NormalTok{(}\AttributeTok{mapping =} \FunctionTok{aes}\NormalTok{(}\AttributeTok{x =}\NormalTok{ rideable\_type, }\AttributeTok{fill =}\NormalTok{ rideable\_type)) }\SpecialCharTok{+} \FunctionTok{facet\_wrap}\NormalTok{(}\SpecialCharTok{\textasciitilde{}}\NormalTok{member\_casual) }\SpecialCharTok{+} \FunctionTok{labs}\NormalTok{(}\AttributeTok{title =} \StringTok{"Bike Share: Ride count between Sep 2021 and Mar 2022"}\NormalTok{) }
\end{Highlighting}
\end{Shaded}

\includegraphics{BikeShare_Analysis_files/figure-latex/unnamed-chunk-10-1.pdf}

Now let's look at date range Apr - Aug

\begin{Shaded}
\begin{Highlighting}[]
\NormalTok{RideCounts\_AprAug }\OtherTok{\textless{}{-}}\NormalTok{ RidesLast12Months }\SpecialCharTok{\%\textgreater{}\%}
  \FunctionTok{select}\NormalTok{(ride\_id, member\_casual, rideable\_type, started\_at) }\SpecialCharTok{\%\textgreater{}\%}
  \FunctionTok{filter}\NormalTok{(started\_at }\SpecialCharTok{\textgreater{}=} \StringTok{\textquotesingle{}2022{-}04{-}01\textquotesingle{}}\NormalTok{, started\_at }\SpecialCharTok{\textless{}=} \StringTok{\textquotesingle{}2022{-}08{-}31\textquotesingle{}}\NormalTok{)}

\FunctionTok{ggplot}\NormalTok{(RideCounts\_AprAug) }\SpecialCharTok{+} \FunctionTok{geom\_bar}\NormalTok{(}\AttributeTok{mapping =} \FunctionTok{aes}\NormalTok{(}\AttributeTok{x =}\NormalTok{ rideable\_type, }\AttributeTok{fill =}\NormalTok{ rideable\_type)) }\SpecialCharTok{+} \FunctionTok{facet\_wrap}\NormalTok{(}\SpecialCharTok{\textasciitilde{}}\NormalTok{member\_casual) }\SpecialCharTok{+} \FunctionTok{labs}\NormalTok{(}\AttributeTok{title =} \StringTok{"Bike Share: Ride count between Apr 2022 and Aug 2022"}\NormalTok{)}
\end{Highlighting}
\end{Shaded}

\includegraphics{BikeShare_Analysis_files/figure-latex/unnamed-chunk-11-1.pdf}

No specific insights we could find from these, although we can still
notice that casual riders mostly prefer to ride electric bikes compared
to member riders.

\hypertarget{insights}{%
\subsection{Insights}\label{insights}}

The Bike share data was prepared, processed and analyzed to derive the
insights of casual riders and member riders. From the analysis, we found
that member rides and casual rides are 58\% and 42\% of total ride
counts. Member riders prefer classic bikes more compared to casual
riders.

Casual riders' average ride duration is high compared to member riders
which could be due to the casual riders mostly take routes between long
distance ride stations. Member riders most frequently ride between
stations include Ellis Ave \& 60Th St, University Ave \& 57th St,
Calumet Ave \& 33Rd St, and State St \& 33Rd St.~The casual riders
between these stations are also quite high, but not as high as the
member riders. The data is not sufficient to know if these casual riders
are same returning customers. More data is needed to confirm this. If
this is true, these casual customers could be converted to member
riders. Most casual ride counts are between stations Streeter Dr \&
Grand Ave, Dusable Lake Shore Dr \& Monroe St, Millennium Park, and
Michigan Ave \& Oak St.

Casual riders prefer both classic bikes and electric bikes equally but
slightly more preference towards classic bikes. They also use docked
bikes, but the docked bike rides are significantly less. This could be
due to less number of docking stations, but this data is not sufficient
in this regard to make an inference.

Looking at the ride counts distribution in the last 12 months for any
seasonality, it is found that the ride count during winter is
significantly low compared to summer. This is true for both casual rides
and member rides. Hence there is no rider type specific seasonality. But
we found that there is a small dip in casual ride count, after the
initial peak in summer but member rides are steady throughout summer
until September after the initial peak in start of summer.

\hypertarget{conclusion-and-recommendations}{%
\subsection{Conclusion and
Recommendations}\label{conclusion-and-recommendations}}

\hypertarget{include-a-six-month-membership-plan-along-with-the-one-year-plan}{%
\paragraph{Include a six month membership plan along with the one year
plan}\label{include-a-six-month-membership-plan-along-with-the-one-year-plan}}

Most of our member riders ride between stations Ellis Ave \& 60Th St,
University Ave \& 57th St, Calumet Ave \& 33Rd St, and State St \& 33Rd
St.~There are significant number of casual rides between these stations
although not same number of rides as member rides. This could be due to
many of the riders preferring not to have a membership because they only
use it during the summer.

\hypertarget{include-more-electric-bikes}{%
\paragraph{Include more electric
bikes}\label{include-more-electric-bikes}}

It is found that casual ride duration are much higher compared to member
rides. It could be hard for them to use classic bikes from far stations
on a daily basis in their routes. As we can see that our casual riders
have opted electric bikes and classic bikes almost equally. If more
electric bikes are available consistently, these riders will prefer to
join as members. But more investigation is needed in this as bike count
by bike type is not available.

\end{document}
